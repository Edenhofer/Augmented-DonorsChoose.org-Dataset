\documentclass[aspectratio=169]{beamer}
\usepackage[utf8]{inputenc}  % Required for umlauts
\usepackage[english]{babel}  % Language
%\usepackage[sfdefault]{roboto}  % Enable sans serif font roboto
%\usepackage{libertine}  % Enable this on Windows to allow for microtype
\usepackage[T1]{fontenc}  % Required for output of umlauts in PDF

\usepackage{mathtools}  % Required for formulas

\usepackage{caption}  % Customize caption aesthetics
\usepackage{tcolorbox}  % Fancy colored boxes
\usepackage{color}
\usepackage{xcolor}  % Highlighting
\usepackage{soul}

\usepackage{booktabs}  % Using pandas' LaTeX output
\usepackage{multirow}  % Enable fancy table structure

\usepackage{listings}  % Insert programming code
\usepackage{graphicx}  % Required to insert images
\usepackage{subcaption}  % Enable sub-figure
\usepackage[space]{grffile}  % Insert images baring a filename which contains spaces
\usepackage{float}  % Allow to forcefully set the location of an object

\usepackage[tracking=true]{microtype} % Required to change character spacing

\usepackage[backend=biber,autocite=footnote,style=authortitle-comp,sorting=none,giveninits=true,doi=false,isbn=false,url=false,eprint=false]{biblatex}
\usepackage{csquotes}  % Ensure proper quotation of texts with babel and polyglossia with biblatex
\usepackage{hyperref}  % Insert clickable references

\usepackage{datetime}  % Flexible date specification

\usepackage{geometry}
\usepackage{scrextend}  % Allow arbitrary indentation

\usepackage{appendixnumberbeamer}  % Fancy page numbering excluding the appendix

% Compile notes into a separate file readable by pdfpc using a custom package which overwrite the `note` macro
\usepackage{../pdfpcnotes}

\addbibresource{../literature.bib}
\renewcommand{\footnotesize}{\tiny}
%\renewcommand*{\bibfont}{\scriptsize}

\newcommand{\leadingzero}[1]{\ifnum#1<10 0\the#1\else\the#1\fi}
\newcommand{\todayddmmyyyy}{\leadingzero{\day}.\leadingzero{\month}.\the\year}
\newcommand{\mathcolorbox}[2]{\colorbox{#1}{$\displaystyle #2$}}

\makeatletter
% Fix subfig in beamer style presentation
\let\@@magyar@captionfix\relax

% Insert [short title] for \section in ToC
\patchcmd{\beamer@section}{{#2}{\the\c@page}}{{#1}{\the\c@page}}{}{}
% Insert [short title] for \section in Navigation
\patchcmd{\beamer@section}{{\the\c@section}{\secname}}{{\the\c@section}{#1}}{}{}
% Insert [short title] for \subsection in ToC
\patchcmd{\beamer@subsection}{{#2}{\the\c@page}}{{#1}{\the\c@page}}{}{}
% Insert [short title] for \subsection in Navigation
\patchcmd{\beamer@subsection}{{\the\c@subsection}{#2}}{{\the\c@subsection}{#1}}{}{}
\makeatother

\setbeamercolor{title}{fg=orange}
\setbeamertemplate{title}{
	\color{orange}
	\textbf{\inserttitle}
}
\setbeamercolor{tableofcontents}{fg=orange}
\setbeamercolor{section in toc}{fg=black}
\setbeamercolor{subsection in toc}{fg=black}
\setbeamertemplate{frametitle}{
	\nointerlineskip%

	\begin{beamercolorbox}[ht=4em,sep=1em,center,wd=\paperwidth]{frametitle}
		\color{orange}
		\bfseries%
		\strut\insertframetitle\strut%
		\mdseries%
		\\
		\small
		\strut\insertframesubtitle\strut
	\end{beamercolorbox}
}
\setbeamertemplate{footline}[text line]{
	\parbox{\linewidth}{
		\tiny
		\color{gray}
		AMIR 2019
		\hfill
		Gordian (\href{mailto:gordian.edenhofer@gmail.com}{gordian.edenhofer@gmail.com})
		\hfill
		\insertframenumber/\inserttotalframenumber%
	}
}
\setbeamertemplate{navigation symbols}{}
\setbeamertemplate{itemize item}{\color{black}$\bullet$}
\setbeamertemplate{itemize subitem}{\color{black}$\circ$}
\setbeamercolor{block title}{fg=black}
\captionsetup{font=scriptsize,labelfont={bf,scriptsize}}

% Customize code blocks
\lstset{basicstyle=\ttfamily,breaklines=true,showstringspaces=false,commentstyle=\color{red},keywordstyle=\color{blue}}

\title{Augmenting the DonorsChoose.org Corpus for Meta-Learning}
\subtitle{``Meta-Learning for Recommender Systems''}
\author[Edenhofer]{\href{mailto:gordian.edenhofer@gmail.com}{Gordian Edenhofer}}
\institute[NII]{
	Working Group of Prof.~Dr.~Beel, Trinity College Dublin \\
	Laboratory of Prof.~Dr.~Akiko~Aizawa, Nationa Institute of Informatics
}
\date[AMIR 2019]{AMIR, \formatdate{14}{04}{2019}}
\subject{Natural Language Processing and Machine Translation}

\begin{document}

\pagenumbering{arabic}

\begin{frame}[plain]  % NOTE: Adding "noframenumbering" as argument confuses `pdfpc`; see https://github.com/pdfpc/pdfpc/issues/367
	\titlepage%
\end{frame}

\section[Introduction]{Research Problem}
\frame{\vfill\centering\tableofcontents[sectionstyle=show/shaded,subsectionstyle=show/hide]\vfill}

% TODO: Some advise for the presentation at AMIR: It would be good if you could somehow address the scope of the workshop, i.e. don't argue "we need more detasets for meta learning", but argue "we need to explore more meta learning in the field of recommender systems, and in this field there are few suitable datasets. If you have time, you might also adjust this in the paper a bit.

\subsection{Lack of Datasets}
\begin{frame}
	\frametitle{\insertsection}
	\framesubtitle{\insertsubsection}

	\begin{itemize}
		\item Meta-learning for recommender systems
		\begin{itemize}
			\item Combine strengths of individual learners
			\item Incorporate more transaction details
		\end{itemize}
		\item Absence of recommendation datasets with metadata
		\begin{itemize}
			\item Time-consuming and cumbersome to construct and implement learning subsystem
			\item Drains resources from actual meta-learning experiment
			\item Manual augmentation is often redundant~\autocite{CUNHA2018128,DBLP:journals/corr/abs-1805-12118,Ekstrand:2012:RFP:2365952.2366002}
		\end{itemize}
	\end{itemize}

	\note{Cumbersome in the sense of, e.g., choice of metric}
	\note{Redundant augmentation as 5 out of 7 reviewed use MovieLense data}
\end{frame}

\subsection{Augmenting DonorsChoose.org}
\begin{frame}
	\frametitle{\insertsection}
	\framesubtitle{\insertsubsection}

	\begin{itemize}
		\item Augment dataset
		\begin{itemize}
			\item Find recommendation dataset suitable for meta-learning
			\item Ensure sufficient user-, item- and transaction-information
			\item Add metadata of single recommender algorithms, statistics, etc.~to the dataset
		\end{itemize}
		\item Encourage evaluation of meta-learner experiments on this dataset
	\end{itemize}

	\note{Allow others to skip augmentation of proper dataset}
\end{frame}

\section[Methodology]{Augmentation}
\frame{\vfill\centering\tableofcontents[sectionstyle=show/shaded,subsectionstyle=show/hide]\vfill}

\subsection{The DonorsChoose.org Dataset}
\begin{frame}
	\frametitle{\insertsection}
	\framesubtitle{\insertsubsection}

	\begin{itemize}
		\item DonorsChoose.org transaction dataset
		\begin{itemize}
			\item \$686M raised
			\item 4,69M transactions (donations); 2,12M users (donors); 1,11M items (projects)
		\end{itemize}
		\item Available features
		\begin{itemize}
			\item User: location, teacher status
			\item Item: location, category, descriptions, funding goals, creator
			\item Transaction: strength (donated amount), time
		\end{itemize}
		\item Augmentation
		\begin{itemize}
			\item Learning subsystem: two collaborative \& two content-based recommender algorithms
			\item Statistics: aggregated user and item statistics
			\item Exemplary meta-learners: algorithm selection problem via stacking hybrid ensemble
		\end{itemize}
	\end{itemize}

	\note{metadata-rich dataset is important for good recommendation}
	\note{DonorsChoose.org corpus a perfect fit due to instrinsic topic-match}
\end{frame}

\begin{frame}
	\frametitle{\insertsection}
	\framesubtitle{\insertsubsection}

	\begin{figure}
		\centering
		\includegraphics[width=\textwidth,height=0.6\textheight,keepaspectratio]{{{../res/Illustrative example of the overall design of the augmented transaction table}}}
		\caption{Illustrative example of the overall design of the augmented transaction table with
		amended learning subsystem performance scores, statistics and meta-learner information.}
	\end{figure}

	\note{plain data to the left; augmented to the right}
\end{frame}

\subsection{Dataset Preparation}
\begin{frame}
	\frametitle{\insertsection}
	\framesubtitle{\insertsubsection}

	\begin{itemize}
		\item House keeping
		\begin{itemize}
			\item Merging duplicate transaction
			\item Remove stop-words
			\item Drop transactions with NaN as location
		\end{itemize}
		\item Requiring users to have donated at least twice (training and testing)
		\item Transformation
		\begin{itemize}
			\item DonationAmount to interaction strength
			\item Sampling 100,000 transaction
		\end{itemize}
	\end{itemize}

	\note{preparation overall preserves a little more than half of the data}
	\note{requiring users to donate twice reduces number of transactions but is necessary}
	\note{conversion to interaction strength incorporates outliers}
\end{frame}

\subsection{Learning Subsystem}
\begin{frame}
	\frametitle{\insertsection}
	\framesubtitle{\insertsubsection}

	\begin{itemize}
		\item Cross-validation using $5$ folds
		\item Top-N accuracy using the recall~\autocite{Cremonesi:2010:PRA:1864708.1864721}
		\begin{itemize}
			\item Randomized subset of $100$ projects the user has not donated to
			\item One project to which the user has donated to
			\item Recall@N is predicted position of donated project in the subset of $101$ projects
		\end{itemize}
	\end{itemize}

	\begin{table}
		\scriptsize
		\centering
		\begin{tabular}{l|lll}
			Filtering method & Algorithm & Computational method & Yields \\
			\hline
			\hline
			collaborative & KNN & nearest-neighbor (2-norm) & RMSE, MAE, \colorbox{yellow}{Recall@N} \\
			& SVD & matrix factorization & RMSE, MAE, \colorbox{yellow}{Recall@N} \\
			\hline
			content-based & TF-IDF & term frequency & cosine similarity, \colorbox{yellow}{Recall@N} \\
			\hline
			content-based, neural network & fastText & word-embedding & cosine similarity, \colorbox{yellow}{Recall@N} \\
		\end{tabular}
		\caption{Architecture of the learning subsystem with a list of properties for each algorithm.}
	\end{table}

	\note{top-n instead of a metric ensures comparability of results}
\end{frame}

\subsection{Statistics and Application of Learning Subsystem}
\begin{frame}
	\frametitle{\insertsection}
	\framesubtitle{\insertsubsection}

	\begin{itemize}
		\item Holdout-validation using $80\%$ train-test-split
		\item Statistics
		\begin{itemize}
			\item Aggregated user information on value counts for past donations
			\item Aggregated item location, category, etc.~information grouped by user
		\end{itemize}
		\item Meta-learner approaches
		\begin{itemize}
			\item Transaction classification via a decision tree
			\item Error prediction using gradient boosting
			\item User-clustering using K-Means
			\item Stacking decision tree as classifier
		\end{itemize}
	\end{itemize}

	\note{easy to compute user-statistics based on train set}
	\note{valueable features for meta-learner}
\end{frame}

\section[Results]{Augmented Corpus}
\frame{\vfill\centering\tableofcontents[sectionstyle=show/shaded,subsectionstyle=show/hide]\vfill}

\subsection{Shape}
\begin{frame}
	\frametitle{\insertsection}
	\framesubtitle{\insertsubsection}

	\begin{itemize}
		\item More than half of transactions preserved ($>>$ sample size)
		\item Sample of 100,000 transactions: 21,087 unique users and 91,827 unique projects
	\end{itemize}

	\begin{figure}[ht]
		\centering
		\subcaptionbox{Transactions per user\label{fig:dist_number_of_interactions_per_user}}{
		\includegraphics[width=0.35\textwidth,height=\textheight,keepaspectratio]{{{../res/DonorID - Distribution of number of interactions per user}}}
		}
		\hspace{2em}
		\subcaptionbox{Ratings\label{fig:dist_transaformed_ratings}}{
		\includegraphics[width=0.35\textwidth,height=\textheight,keepaspectratio]{{{../res/DonationAmount - Distribution of ratings for logarithmic bins and excluded outliers}}}
		}
		\caption{Distribution of the number of transactions per user and the rating scores to which the donated amount has been transformed to.}
		\label{fig:dataset_sample_dist}
	\end{figure}
\end{frame}

\subsection{Learning Subsystem}
\begin{frame}
	\frametitle{\insertsection}
	\framesubtitle{\insertsubsection}

	\begin{figure}
		\centering
		\includegraphics[width=\textwidth,height=0.6\textheight,keepaspectratio]{{{../res/Learning subsystem - Position in Top-N test set}}}
		\caption{Statistical features of the distribution of the donated item position in the Recall@N test set. The dashed green lines indicate the means, the straight orange lines the medians.}
	\end{figure}

	\note{most essential result of this work!}
\end{frame}

\subsection{Exemplary Meta-Learning Experiment}
\begin{frame}
	\frametitle{\insertsection}
	\framesubtitle{\insertsubsection}

	\begin{figure}
		\centering
		\includegraphics[width=\textwidth,height=0.6\textheight,keepaspectratio]{{{../res/Meta-learner Performance - Average position in Top-N test set for various meta-learner algorithms with augmented learning subsystem filtering techniques}}}
		\caption{Performance of the four meta-learners in comparison to the overall best algorithm.}
	\end{figure}

	\note{just a small experiment}
\end{frame}

\section{Conclusion}
\frame{\vfill\centering\tableofcontents[sectionstyle=show/shaded,subsectionstyle=show/hide]\vfill}

\subsection{Extensively Augmented DonorsChoose.org Dataset}
\begin{frame}
	\frametitle{\insertsection}
	\framesubtitle{\insertsubsection}

	\begin{itemize}
		\item Feature-rich and big corpus suitable for recommendations
		\item Extensively augmented features
		\begin{itemize}
			\item Learning subsystem
			\item Aggregated user and transaction statistics
			\item First success in solving the Algorithm Selection Problem via switching hybrid ensemble
		\end{itemize}
		{
		\setbeamertemplate{itemize item}{\color{black}$\Rightarrow$}
		\item Excellent candidate for further meta-learning experiments
		}
	\end{itemize}
\end{frame}

\section*{Thank You}
{
	\setbeamertemplate{footline}[text line]{}
	\begin{frame}
		\frametitle{\insertsection}
		\framesubtitle{\insertsubsection}

		\begin{columns}[t]
			\begin{column}{0.175\textwidth}
				\centering
				\includegraphics[width=\textwidth,height=\textheight,keepaspectratio]{{{../res/Profile picture of Gordian}}} \\
				Gordian
			\end{column}

			\begin{column}{0.175\textwidth}
				\centering
				\includegraphics[width=\textwidth,height=\textheight,keepaspectratio]{{{../res/Profile picture of Andrew}}} \\
				Andrew
			\end{column}

			\begin{column}{0.175\textwidth}
				\centering
				\includegraphics[width=\textwidth,height=\textheight,keepaspectratio]{{{../res/Profile picture of Akiko}}} \\
				Aizawa
			\end{column}

			\begin{column}{0.175\textwidth}
				\centering
				\includegraphics[width=\textwidth,height=\textheight,keepaspectratio]{{{../res/Profile picture of Joeran}}} \\
				Beel
			\end{column}
		\end{columns}

		\centering
		\vspace{3em}
		{\Large\inserttitle{}}
		\vspace{1em}
		\bfseries{\href{https://github.com/BeelGroup/Augmented-DonorsChoose.org-Dataset}{GitHub:BeelGroup/Augmented-DonorsChoose.org-Dataset}}
		\vspace{1em}
		\href{mailto:gordian.edenhofer@gmail.com}{gordian.edenhofer@gmail.com}
	\end{frame}
}

\appendix
\section*{Appendix}

\subsection{Meta-Learner}
\begin{frame}
	\frametitle{\insertsection}
	\framesubtitle{\insertsubsection}

	\scalebox{.47}{
	\centering
	% Extract the table via the following command and add the necessary midrule to separate inputs from targets
	% meta_items.loc[train_idx][sorted(feature_columns) + ['SubalgorithmCategory'] + sorted(alg_acc_to_alg_columns.keys())].head(5).transpose().to_latex('meta_items: table head.tex', multirow=True, header=False)
	\input{"../res/meta_items: table head"}
	}
\end{frame}

\subsection{Meta-Learner}
\begin{frame}
	\frametitle{\insertsection}
	\framesubtitle{\insertsubsection}

	\begin{figure}
		\centering
		\subcaptionbox{Error prediction}{
		\includegraphics[width=0.45\textwidth,height=\textheight,keepaspectratio]{{{../res/Meta-Learner Architecture: Error prediction algorithm}}}
		}
		\subcaptionbox{Classification}{
		\includegraphics[width=0.45\textwidth,height=\textheight,keepaspectratio]{{{../res/Meta-Learner Architecture: Classification algorithm}}}
		}
		\caption{Meta-learner architecture for an error prediction- and a classification-approach.}
	\end{figure}
\end{frame}

\end{document}
